\documentclass{article}
\usepackage{preamble}
\input{preamble}
% Redefine \part to have the desired format
\makeatletter
\renewcommand\thepart{\arabic{part}} % Use Arabic numbers for \part
\makeatother
\titleformat{\section}
  {\normalfont\LARGE\bfseries} % Format for numbered sections
  {\thesection.}              % Numbering format
  {1em}                       % Space between number and title
  {}

% Define appearance for \section*
\titleformat{name=\section,numberless}
  {\normalfont\LARGE\bfseries} % Format for unnumbered sections
  {}
  {0pt}
  {}

% Similarly, define for \subsection and \subsection* if needed
\titleformat{\subsection}
  {\normalfont\Large\bfseries}
  {\thesubsection.}
  {1em}
  {}

\titleformat{name=\subsection,numberless}
  {\normalfont\Large\bfseries}
  {}
  {0pt}
  {}

\graphicspath{ {images/} }
%%%%%%%%%%%%%%%% Document %%%%%%%%%%%%%%%%
\begin{document}
\begin{center}
  {\huge \underline{Variational Methods 049064} \\[7pt] Exercise 1}
\end{center}
\part{Analytic Exercises}
\section*{Question 1 (Linear Diffusion)}
Let $u(x,t)$ be a real-valued function. Consider the 1D, unbounded domain, linear diffusion, defined by
\begin{equation} \label{eq:q1PDE}
u_t = u_{xx}, \quad u(x,0) = f(x), \quad t \in [0, \infty)
\end{equation}
where \(t\) is the time variable, \(x\) is the spatial variable, \(u_t\) and \(u_{xx}\) are respectively the time derivative and the second spatial derivative of \(u\), and the function \(f(x)\) is the initial condition.

(a) For a real-valued function \(u \in \mathbb{R}\), show that the solution of the above PDE is a Gaussian convolution with the initial condition, namely
\begin{equation}
  u(x,t) = f(x) * g_{\sigma(t)}(x)
\end{equation}
where $g_{\sigma(t)}$ is the Gaussian kernel.

\textbf{\ul{Solution}:} 

Denote:
\[
  (\mathcal{F} \left\{ u \right\})(s) =   \hat{u}(s,t) = \int_{-\infty}^{\infty} e^{-i 2 \pi s x} u(x,t) \d{x}
\]

We apply the spatial FT to both sides of~\cref{eq:q1PDE}, and get:
\[
  \int_{-\infty}^{\infty} e^{- i 2 \pi s x} u_{t} \d{x} = \int_{-\infty}^{\infty} e^{-i 2\pi s x} u_{xx}   \d{x}
\]


Using the Fourier Transform of the n-th derivative:
\[
  \mathcal{F} \left\{ \frac{d^{(n)}g(x)}{d x^n}  \right\} = (i 2\pi s )^{n} \mathcal{F} \left\{ g \right\}
\]
and the fact that the spatial FT of a time derivative is the time derivative of the spatial FT, we get:
\[
  \frac{\partial }{\partial t} \hat{u}(s,t) = - (2 \pi s)^2  \cdot  \hat{u}(s,t)   
\]
This is a simple ODE wrt to $t$. We solve this  and get:
\[
  \hat{u}(s,t) = c(s) e^{- (2 \pi s)^2 t}  
\]
we plug $t=0$ into the above equation, and get:
\[
  \hat{u}(s,0) = c(s) 
\]
and since
\[
  \hat{u}(s,0) = \int_{-\infty}^{\infty} e^{-i 2 \pi s x} u(x,0) \d{x} = \int_{-\infty}^{\infty} e^{-i 2 \pi s x} f(x) \d{x} = \hat{f}(s)  
\]
we get:
\[
  \hat{u}(s,t) = \hat{f}(s) e^{-(2 \pi s) ^2 t}   
\]
i.e. we notice that the spatial FT of $u$ is a multiplication of two functions. Applying the inverse spatial FT, we get that $u(x,t)$ is a convolution of the inverse FT of those functions, i.e.:
\[
  u(x,t) = f(x) \ast \Big( \mathcal{F}^{-1} \left\{ e^{-4 \pi^2 s ^2 t}  \right\} (x) \Big)
\]
Recalling that:
\[
  \mathcal{F}^{-1} \left\{ e^{- \pi s ^2}  \right\}(x) = e^{- \pi x ^2} 
\]
and the Fourier stretch theorem:
\[
  \big(\mathcal{F} \left\{ f(ax) \right\}\big)(s) = \frac{1}{|a|} \, \hat{f}(\frac{s}{a}) 
\]
we get that:
\[
  \big( \mathcal{F} ^{-1}  \left\{ e^{ - 4 \pi ^2 s ^2 t} \right\} \big) (x) =  \frac{1}{\sqrt{ 4 \pi t }} e^{ - \pi \frac{x ^2}{4 \pi t}}  = \frac{1}{\sqrt{  4 \pi t }} e^{ - \frac{x ^2}{4t}} 
\]
If we denote:
\[
  g_{\sigma (t)}(x) = \frac{1}{\sqrt{ 4 \pi t }} e^{- \frac{x ^2}{4t}} 
\]
we get:
\[
  \boxed  {u(x,t) = f(x) \ast g_{\sigma(t)}(x)}
\]

--------------------------------------------- 

Then, find the relation between the time variable \(t\) and the standard deviation \(\sigma\).

\textbf{\ul{Solution:}}

By the formula:
\[
  g_{\sigma (t)}(x) = \frac{1}{\sqrt{ 4 \pi t }} e^{- \frac{x ^2}{4t}} 
\]
and the standard Gaussian:
\[
  \frac{1}{\sqrt{ 2 \pi \sigma ^2 }} e^{- \frac{x ^2}{2 \sigma ^2}} 
\]
we get that $2t = \sigma ^2$, i.e. $t = \frac{\sigma}{\sqrt{ 2 }}$

---------------------------------------------

(b) Assume the initial condition \(f(x)\) is

\begin{equation}
f(x) = \sin(\omega_1 x) + \sin(\omega_2 x)
\end{equation}
where \(\omega_1, \omega_2\) are positive constants. Use $u(x,t) = f(x) * g_\sigma(t)(x)
 $ to show that \(u(x,t)\) is 

\begin{equation}
u(x,t) = \sin(\omega_1 x)e^{-\omega_1^2 t} + \sin(\omega_2 x)e^{-\omega_2^2 t}.
\end{equation}

\textbf{\ul{Solution:}}

From a Fourier table, we have:
\[
  \mathcal{F} \left\{ \sin(\omega x) \right\} =  \frac{1}{2 i} \Big( \delta(s- \frac{\omega}{2 \pi}) - \delta(s + \frac{\omega}{2 \pi}) \Big) 
\]
From our previous derivations, we have that
\[
  \mathcal{F} \left\{ \frac{1}{\sqrt{ 4 \pi t }} e^{- \frac{x ^2}{4t}}  \right\}  = e^{- 4 \pi ^2 s ^2 t}
\]
Therefore, using the fact that the FT of a convolution is the multiplication of the respective Fourier transforms, we get:
\begin{align*}
  \mathcal{F} \left\{ \sin(\omega x) \ast g_{\sigma(t)}(x)   \right\}  &= \Big( \frac{1}{2 i} \big( \delta(s- \frac{\omega}{2 \pi}) - \delta(s + \frac{\omega}{2 \pi}) \big) 
  \Big) \cdot  \Big( e^{- 4 \pi ^2 s ^2 t}  \Big) \\
  &=  \frac{1}{2 i} \Big( \delta(s- \frac{\omega}{2 \pi}) e^{- 4 \pi ^2 s ^2 t} - \delta(s + \frac{\omega}{2 \pi}) e^{-4 \pi ^2 s ^2 t} \Big) \\
  &= \frac{1}{2 i} \Big( \delta(s-\frac{\omega}{2 \pi}) e^{- \omega ^2 t } - \delta(s + \frac{\omega}{2 \pi}) e^{- \omega ^2 t} \Big) \\
  &= \frac{1 }{2 i} e^{- \omega ^2 t} \Big( \delta(s- \frac{\omega}{2 \pi}) - \delta(s + \frac{\omega}{2 \pi}) \Big) 
\end{align*}
i.e.:
\begin{equation} \label{eq:q1MidDeriv}
  \mathcal{F} \left\{ \sin(\omega x) \ast g_{\sigma(t)}(x)   \right\}  = \frac{1 }{2 i} e^{- \omega ^2 t} \Big( \delta(s- \frac{\omega}{2 \pi}) - \delta(s + \frac{\omega}{2 \pi}) \Big) 
\end{equation}


We wish to apply the inverse FT. We notice $\frac{1 }{2 i} e^{- \omega ^2 t}$ is a constant wrt to $s$, and we recall that:
\[
  \mathcal{F} ^{-1}  \left\{  \delta(s \mp  \frac{\omega}{2 \pi} ) \right\} = e^{\pm  i \omega x} 
\]
and therefore:
\[
  \frac{1}{2 i} \mathcal{F} ^{-1} \Big\{ \delta(s- \frac{\omega}{2 \pi}) - \delta(s + \frac{\omega}{2 \pi})  \Big\} = \frac{1 }{2 i} \big( e^{ i \omega x} - e^{-i \omega x}  \big) = \sin(\omega x)
\]

We apply the inverse Fourier transform on both sides of~\cref{eq:q1MidDeriv}. Using the transform linearity, we get:
\[
  \sin(\omega x) \ast  g_{\sigma(t)}(x) = \sin(\omega x)  e^{-\omega ^2 t} 
\]

Let $f(x) = \sin(\omega_1 x) + \sin(\omega_2x)$.
Convolution is linear, therefore

\begin{align*}
  f(x) \ast g_{\sigma(t)}(x) &= (\sin(\omega_1 x) \ast g_{\sigma(t)}(x) ) + ( \sin(\omega_2 x) \ast g_{\sigma(t)}(x) ) \\
  &= \sin(\omega_1 x)  e^{-\omega_1^2 t}  + \sin(\omega_2 x)  e^{-\omega_2 ^2 t} 
\end{align*}
---------------------------------------------

(c) For the solution:
\begin{equation} \label{eq:q1FSinSol}
  u(x,t) = \sin(\omega_1 x)e^{-\omega_1^2 t} + \sin(\omega_2 x)e^{-\omega_2^2 t}
\end{equation}

(i) Write the explicit backward-difference \ul{in time} expression of \(u(x, \Delta t)\). Keep the spatial coordinates continuous.

(ii) Write the analytic solution of the approximation of \(u(x, \Delta t)\) given above in (i).

\textbf{\ul{Solution:}}

The backward-difference is:
\[
  u_{t}(x,t) = \frac{u(x,t) - u(x,t -\Delta t)}{\Delta t}
\]
Therefore, we get that:
\begin{align*}
  & \frac{u(x,t) - u(x,t -\Delta t)}{\Delta t} = u_{xx}(x, t) \\
  \Rightarrow \ & u(x,t) = u(x,t-\Delta t) + \Delta t \, u_{xx}(x,t)
\end{align*}
at $t= \Delta t$, we get:
\begin{align*}
  u(x,\Delta t) &= u(x,0) + \Delta t \, u_{xx}(x, \Delta t) = f(x) + \Delta t \, u_{xx}(x,\Delta t) \\
\end{align*}
We compute $u_{xx}(x,t)$ from the solution at~\cref{eq:q1FSinSol}, and get:
\[
  u_{xx}(x,t) = -\omega_1 ^2 \sin(\omega_1 x) e^{-\omega_1 ^2 t} - \omega_2 ^2 \sin(\omega_2 x) e^{- \omega_2 ^2 t}  
\]
we plug the above into our expression of $u(x, \Delta t)$ and we get:
\begin{align*}
  u(x, \Delta t) &= \sin(\omega_1x) + \sin(\omega_2x)  + \Delta t \cdot \big( -\omega_1 ^2 \sin(\omega_1 x) e^{-\omega_1 ^2 \Delta t} - \omega_2 ^2 \sin(\omega_2 x) e^{- \omega_2 ^2 \Delta t} \big) \\
  &= \sin(\omega_1 x) \big( 1 - \Delta t \omega_1 ^2 e^{-\omega_1^2 \Delta t}  \big) + \sin(\omega_2 x) \big( 1 - \Delta t \omega_2 ^2 e^{-\omega_2^2 \Delta t}  \big)
\end{align*}

 ---------------------------------------------

(iii) Show that the extremum principle is kept under some condition. What is the condition? Is it similar to a stability condition learned in class? Explain.

\ul{\textbf{Solution}:}

The initial condition is:
\[
  u(x,0) = f(x) = \sin(\omega_1 x) e^{-\omega_1 ^2 t} + \sin(\omega_2 x) e^{-\omega_2 ^2 t}  
\]
Since $\omega_i > 0$, we have that  $-2 \leq f(x) \leq 2$

We have two modes $\sin(\omega_i x)$, with their amplitudes being:
\[
  1-\Delta t \omega_i ^2 e^{- \omega_i ^2 \Delta t} 
\]
We shall examine the amplitudes of the modes, and determine whether they violate the maximum principle, creating new maxima. 


Since $\omega_i > 0$ and $\Delta t > 0$, we have that $\Delta t \omega_i ^2 > 0$ and also we have that $e^{-\omega_i ^2 \Delta t} > 0$, therefore:
\begin{align*}
  & \Delta t w_i ^2 e^{- \omega_i ^2 \Delta t} > 0 \\
\Rightarrow \ & - \Delta t \omega_i ^2 e^{- \omega_i ^2 \Delta t} < 0 \\
  \Rightarrow \ & 1 - \Delta t \omega_i ^2 e^{- \omega_i ^2 \Delta t} < 1
\end{align*}

define:
\[
  h(\Delta t) = 1- \Delta t \omega_i ^2 e^{-\omega_i ^2 \Delta t} 
\]
differentiating, we get that $h$ has a minimum at $\Delta t = \frac{1}{\omega_i ^2}$, and its value is $h(\Delta t = \frac{1}{\omega_i ^2}) = 1-e^{-1}$. 

Therefore:
\begin{align*}
0 < 1- e^{-1} \leq 1- \Delta t \omega_i ^2 e^{-\omega_i ^2 \Delta t}  < 1
\end{align*}


Therefore, after one iteration, the amplitude of the modes decreases, and we have:
\[
  -2 < u(x,\Delta t) < 2
\]

Therefore, assuming that $u$ is bounded and decays at infinity, it satisfies the maximum principle, and no new maxima are created.

This is similar to the CFL stability condition learned in class, where we've ensured that the spatial and temporal discretization steps are such that no new maxima is accidentally created.
\end{document}

