\documentclass{article}
\usepackage{preamble}
\input{preamble}
% Redefine \part to have the desired format
\makeatletter
\renewcommand\thepart{\arabic{part}} % Use Arabic numbers for \part
\makeatother
\titleformat{\section}
  {\normalfont\LARGE\bfseries} % Format for numbered sections
  {\thesection.}              % Numbering format
  {1em}                       % Space between number and title
  {}

% Define appearance for \section*
\titleformat{name=\section,numberless}
  {\normalfont\LARGE\bfseries} % Format for unnumbered sections
  {}
  {0pt}
  {}

% Similarly, define for \subsection and \subsection* if needed
\titleformat{\subsection}
  {\normalfont\Large\bfseries}
  {\thesubsection.}
  {1em}
  {}

\titleformat{name=\subsection,numberless}
  {\normalfont\Large\bfseries}
  {}
  {0pt}
  {}

\graphicspath{ {images/} }
%%%%%%%%%%%%%%%% Document %%%%%%%%%%%%%%%%
\begin{document}
\begin{center}
  {\huge \underline{Variational Methods 049064} \\[7pt] Exercise 1}
\end{center}
\part{Analytic Exercises}
\section*{Question 1 (Linear Diffusion)}
Let $u(x,t)$ be a real-valued function. Consider the 1D, unbounded domain, linear diffusion, defined by
\begin{equation} \label{eq:q1PDE}
u_t = u_{xx}, \quad u(x,0) = f(x), \quad t \in [0, \infty)
\end{equation}
where \(t\) is the time variable, \(x\) is the spatial variable, \(u_t\) and \(u_{xx}\) are respectively the time derivative and the second spatial derivative of \(u\), and the function \(f(x)\) is the initial condition.

(a) For a real-valued function \(u \in \mathbb{R}\), show that the solution of the above PDE is a Gaussian convolution with the initial condition, namely
\begin{equation}
  u(x,t) = f(x) * g_{\sigma(t)}(x)
\end{equation}
where $g_{\sigma(t)} = a$ is the Gaussian kernel.

\textbf{\ul{Solution}:} 

Denote:
\[
  (\mathcal{F} \left\{ u \right\})(s) =   \hat{u}(s,t) = \int_{-\infty}^{\infty} e^{-i 2 \pi s x} u(x,t) \d{x}
\]

We apply the spatial FT to both sides of~\cref{eq:q1PDE}, and get:
\[
  \int_{-\infty}^{\infty} e^{- i 2 \pi s x} u_{t} \d{x} = \int_{-\infty}^{\infty} e^{-i 2\pi s x} u_{xx}   \d{x}
\]


Using the Fourier Transform of the n-th derivative:
\[
  \mathcal{F} \left\{ \frac{d^{(n)}g(x)}{d x^n}  \right\} = (i 2\pi s )^{n} \mathcal{F} \left\{ g \right\}
\]
and the fact that the spatial FT of a time derivative is the time derivative of the spatial FT, we get:
\[
  \frac{\partial }{\partial t} \hat{u}(s,t) = - (2 \pi s)^2  \cdot  \hat{u}(s,t)   
\]
This is a simple ODE wrt to $t$. We solve this  and get:
\[
  \hat{u}(s,t) = c(s) e^{- (2 \pi s)^2 t}  
\]
we plug $t=0$ into the above equation, and get:
\[
  \hat{u}(s,0) = c(s) 
\]
and since
\[
  \hat{u}(s,0) = \int_{-\infty}^{\infty} e^{-i 2 \pi s x} u(x,0) \d{x} = \int_{-\infty}^{\infty} e^{-i 2 \pi s x} f(x) \d{x} = \hat{f}(s)  
\]
we get:
\[
  \hat{u}(s,t) = \hat{f}(s) e^{-(2 \pi s) ^2 t}   
\]
i.e. we notice that the spatial FT of $u$ is a multiplication of two functions. Applying the inverse spatial FT, we get that $u(x,t)$ is a convolution of the inverse FT of those functions, i.e.:
\[
  u(x,t) = f(x) \ast \Big( \mathcal{F}^{-1} \left\{ e^{-4 \pi^2 s ^2 t}  \right\} (x) \Big)
\]
Recalling that:
\[
  \mathcal{F}^{-1} \left\{ e^{- \pi s ^2}  \right\}(x) = e^{- \pi x ^2} 
\]
and the Fourier stretch theorem:
\[
  \big(\mathcal{F} \left\{ f(ax) \right\}\big)(s) = \frac{1}{|a|} \, \hat{f}(\frac{s}{a}) 
\]
we get that:
\[
  \big( \mathcal{F} ^{-1}  \left\{ e^{ - 4 \pi ^2 s ^2 t} \right\} \big) (x) =  \frac{1}{\sqrt{ 4 \pi t }} e^{ - \pi \frac{x ^2}{4 \pi t}}  = \frac{1}{\sqrt{  4 \pi t }} e^{ - \frac{x ^2}{4t}} 
\]
If we denote:
\[
  g_{\sigma (t)}(x) = \frac{1}{\sqrt{ 4 \pi t }} e^{- \frac{x ^2}{4t}} 
\]
we get:
\[
  \boxed  {u(x,t) = f(x) \ast g_{\sigma(t)}(x)}
\]

--------------------------------------------- 

Then, find the relation between the time variable \(t\) and the standard deviation \(\sigma\).

\textbf{\ul{Solution:}}

By the formula:
\[
  g_{\sigma (t)}(x) = \frac{1}{\sqrt{ 4 \pi t }} e^{- \frac{x ^2}{4t}} 
\]
We notice that 
---------------------------------------------
(b) Assume the initial condition \(f(x)\) is

\begin{equation}
f(x) = \sin(\omega_1 x) + \sin(\omega_2 x)
\end{equation}

where \(\omega_1, \omega_2\) are positive constants. Use $u(x,t) = f(x) * g_\sigma(t)(x)
 $ to show that \(u(t, x)\) is 

\begin{equation}
u(x,t) = \sin(\omega_1 x)e^{-\omega_1^2 t} + \sin(\omega_2 x)e^{-\omega_2^2 t}.
\end{equation}

(c) For the solution:
\[
  u(x,t) = \sin(\omega_1 x)e^{-\omega_1^2 t} + \sin(\omega_2 x)e^{-\omega_2^2 t}.
\]

1. Write the explicit backward-difference in time expression of \(u(x, \Delta t)\). Keep the spatial coordinates continuous.

2. Write the analytic solution of the approximation of \(u(x, \Delta t)\) given above in (i).

3. Show that the extremum principle is kept under some condition. What is the condition? Is it similar to a stability condition learned in class? Explain.





Let $u(x,t): \mathbb{R} \times [0, \infty] \to \mathbb{R}$ denote the temperature at position $x$ at time $t$. Solve the IVP:
\begin{equation} \label{eq:unbounded1DHeatEqn}
  u_{t} = u_{xx} \quad, u(x,t=0) = f(x)
\end{equation}
where $f$ is the initial condition. 

\begin{enumerate}[label=\alph*), ref=\alph*]
\item \ul{Claim}: The solution of~\cref{eq:unbounded1DHeatEqn} is a Gaussian convolution with the initial condition, namely:
\[
  u(x,t) = f(x) \ast g_{\sigma(t)} (x)
\]
where $g_{\sigma(t)}$ is the Gaussian Kernel. 

\begin{proof}
a
\end{proof}

  \item 
\end{enumerate}

\end{document}

